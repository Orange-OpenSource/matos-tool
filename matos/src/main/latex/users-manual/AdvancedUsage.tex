\chapter{Customization of security profiles} \label{AdvancedUsage}


\section{Analysis Profiles Specification}\label{ProfFormalization}
In this section, we present what can be specified in a profile and the
syntax to do so.

\subsection*{WARNING} 
\begin{itemize}
\item France Telecom publishes the specification of
  analysis profiles for your convenience. However, be advised that this
  specification may change in the future. 
\item If you change your analysis profilesse keep in mind that
  profiles widely condition the behaviour of the tool's
  engine. Therefore, any change should be carefully understood and
  validated prior to any operational usage. \textbf{France Telecom rejects any
  responsibility in case of problems resulting from the bad
  interpretation of the specification's semantics}. Cf. also
  \ref{AnaProfs}. However, we can provide assistance in this task and
  do the changes for you.
\end{itemize}

\subsection{How Are Profiles Stored?}
Analysis profiles are XML files that must be placed in
\texttt{lib/definitions} folder of your installation directory on
Windows (\texttt{lib/matos/definitions} folder on Linux).

\subsection{What's In a Profile?}
A profile basically lets you define 4 things:
\begin{itemize}
\item \textbf{Header}: General information on the profile.
\item \textbf{Options}: To overload options set by default in the tool's
configuration.
\item \textbf{Rules}: To define the Java methods to be watched
during the Features Usage phase.
\item \textbf{Reports}: To associate messages to some result values (what messages
should be raised according to the results obtained).
\end{itemize}

\subsection{Syntax by the Example}\label{ProfSyntax}

\subsubsection{Header}
A profile typically starts as shown below. The two first lines are
mandatory and must be written as shown. In the \texttt{<matos>}
tag, you're required to specify at least a name for the profile.
The \texttt{</matos>} closing tag must appear at the very end of
the file. Your entire profile data will be specified inside the
\texttt{<matos>} element.

\begin{verbatim}
  <?xml version='1.0' encoding='utf-8'?>
  <!DOCTYPE matos SYSTEM "../matos.dtd">

  <matos name="MyProfile"
         description="The purpose of the present profile is..."
         version="1.5"
         date="18.08.2006">
\end{verbatim}

\subsubsection{Options} \label{optionsSecurityProfile}
Configuration options may be set for a profile. The options you can
set are those presented in \ref{AdvConfigOptions}. Values set in a
profile take precedence over those set as defaults in the configuration
file.
\begin{verbatim}
  <option name="anasoot.treatConcatenation" value="false">
      <description>
        This option controls if a special treatment is applied to the
        <code>StringBuffer.append</code> method. In this profile, string 
        concatenation is not treated.
      </description>
  </option>
\end{verbatim}

\subsubsection{Syntax of method signatures}
We often need to specify methods. For this, we use the signature  of the method.
The signature is the description of the name and types of arguments and
result. The syntax is the following:
\begin{verbatim}
resultType methodName(argType1,...,argTypen)
\end{verbatim}
Note that the number of spaces is significant and that you can only put one
space after the result type. Types are fully qualified names for classes and
are the class name followed by the correct number of [] for array types. 
The name of primitive type is the regular java name.

\subsubsection{Rules (probes)}
A rule defines a particular observation target for the Features Usage
phase, namely a method identifier plus the arguments of that method to
be observed. Conceptually, a rule can be seen as a probe on the
defined target.

In the following example, we define a rule that will check how the
MIDlet uses the network connection feature, with a focus on the
address used in the connection. This address is passed through the
argument number 1 of the method. Values returned by this rule will be
reported using the report called \texttt{net}. The description element
is always optional.
\begin{verbatim}
 rule name="connection">
  <description>
    This rule is to observe what addresses are used when the
    midlet tries to open a network connection.
  </description>
  <args class="javax.microedition.io.Connector"
    signature="javax.microedition.io.Connection open(java.lang.String)"
    report="net">
     <argument position="1"/>
  </args>
</rule>
\end{verbatim}

In fact there are three types of rules:
\begin{itemize}
\item{\textbf{Observation of method argument(s)}}: this is the above
  example. Note that the fully-qualified identifier of the target
  method must be given (absolute package path expected for all types).
  Multiple \texttt{argument} elements can be specified if
  multiple arguments are to be observed. Arguments are numbered
  starting from 1, from left to right. 
% TODO: COMPRENDRE PUIS TRADUIRE: La position 0 correspond � l'objet
% racine de l'appel (\texttt{this}).
\item{\textbf{Observation of a class field}}: in the following
  example, we define a probe on the field \texttt{name} of the class
  \texttt{Person}. Here again, types must be fully-qualified.
\begin{verbatim}
  <rule name="peoplesName">
    <field class="myPackage.Person" type="String name"/>   
  </rule>
\end{verbatim}
\item{\textbf{Observation of a method's return value}}: in the
  following example, we define a probe that provides the values
  returned by the method that gives the name of a record
  store.
\begin{verbatim}
  <rule name="recordStoreName">
    <return class="javax.microedition.rms.RecordStore" 
            type="java.lang.String getName()"/>   
  </rule>
\end{verbatim}
\end{itemize}

\subsubsection{Reports}
As seen above, each rule is associated to a report element. The report
defines what messages will be produced according to those
results. 
Reports are identified by names. There can be only one report with a
given name and it is the first defined that takes precedence.
Besides the name attribute, there is an optional alias attributes that
can defines a comma separated list of alternate names. This is useful
to override a list of automatically generated reports with 
one defined by hand.

There are three types of report:

\begin{itemize}
\item{\textbf{Pseudo constant report}}: a report that offers a choice
  among 2 bodies (messages). If ALL possible values of the observed data -- the
  probe's target -- turned out to be fully determined, the
  \texttt{pass} body is displayed. If at least in one case, the value
  failed to be fully computed, then the \texttt{fail} message is
  produced.
\begin{verbatim}
  <report name="R1">
    <pseudoConstant>
      <fail> 
        In some cases, the argument(s) passed to method %C.%M
        can't be fully determined.
      </fail>
      <pass> 
        Managed to compute <b>all</b> possible values of observed 
        argument(s). 
      </pass>
    </pseudoConstant>
  </report>
\end{verbatim}
Each message is actually a piece of HTML code (notice the \texttt{<b>}
bold-face tag in the example above). Besides,  note that we use some
additional non-HTML tags: \texttt{\%C} and \texttt{\%M}. They are
interpreted by the report generator, and are explained further;
however, these special tags can't be used in the \texttt{pass} message
part.

\item{\textbf{Pseudo string report}}: a report that offers a choice
  among N messages. Each message is associated to a filter, and for
  each value computed, filters are evaluated one by one (following the
  order in which they are declared), until the value matches one of
  them, in which case the associated message is displayed. Filters are
  regular expressions (patterns).
\begin{verbatim}
  <report name="netS">
    <description>
      This report extracts information on the schemes used in the URL.
    </description>
    <pseudoString>
      <filter pattern="\x22([^:]*):([^\x22]*)\x22">
        <b>The application opens a network connection:</b>
        <ul> 
          <li> Method called: %C.%M </li>
          <li> Method hosting the call: %c.%m </li>
          <li> URL used: <b>%r</b> </li>
          <li> Connection type: <b>%1</b> </li>
        </ul>
      </filter>
      <filter pattern=".*">
        <b>The application opens a network connection (but of
           an <font color="red">unknown type</font>):</b>
        <ul> 
          <li> Method called: %C.%M </li>
          <li> Method hosting the call: %c.%m </li>
          <li> URL used: <b>%r</b> </li>
        </ul>
      </filter>
    </pseudoString>
  </report>
\end{verbatim}
Note that the second filter here is associated to the \texttt{''.*''}
expression that means ``anything else''. A special filter called \texttt{default}
also means ``anything else'', but is produced only once even if many
values can't be matched by any filter above, while the \texttt{''.*''}
case is produced for every such value. 
\begin{verbatim}
    ../..
    <pseudoString>
      <filter pattern="\x22([^:]*):([^\x22]*)\x22">
         (The message body)
      </filter>
      <filter ../..>

      <default> Oops... no expression matched in report R1! </default>
    </pseudoString>
\end{verbatim}
Filters are now named. When one use the XML output, the result will only
contain the name of the filter that matched.
\item{\textbf{Simple message}}: a report that just print a message. But this
message may refer to the argument. For example to print all the potential values
of this argument without filtering. The only useful macros are therefore 
  \begin{itemize}
    \item \texttt{\%r} the complete value
    \item \texttt{\%C}, \texttt{\%M} and \texttt{\%S} the specification of the
    method called.
    \item the specification of the caller method.
  \end{itemize}
\begin{verbatim}
     <message>The list of objects used : %r </message>
\end{verbatim}
  You can put simple HTML in the message definition. If HTML is not
  well-balanced XML, you must escape the HTML entities (replace $<$ with
  \verb!&lt;!)
\item{\textbf{Forbidden method message}}: a report that prints a message when a
forbidden method is used. 
  \begin{itemize}
    \item \texttt{\%p} the list of program points where it is used.
  \end{itemize}
\begin{verbatim}
     <message>The method is called at the following points: %p
     </message>
\end{verbatim}
  
\item{\textbf{Conjunction report}}: a report that compiles the output
  of two other reports. The first report is executed, then the second
  one. However, if the first one is a pseudo constant
  report that fails, the second report is not produced. The reason for
  this behaviour is that it makes no sense to perform pattern matching
  on a value if we failed to fully establish that value.

\begin{verbatim}
  <report name="net">
    <conjunction>
      <ref name = "R1"/>
      <ref name = "R2"/>
    </conjunction>
  </report>
\end{verbatim}
    
  Rules may give back results on several arguments but report should only refer
  to one argument at a time. This is the only report where you can specify the
  position of the argument that must be checked (as a number between 0 and the
  number of arguments checked by the rule).
  
  If conjunction reports are included into other conjunction report, the
  position should usually only be specified in the first. The element is then
  immediately filtered.
  
  For example if you have a rule that check two arguments you can build two
  filters \texttt{F0} and \texttt{F1} for each argument and a simple message
  report \texttt{M} to print information on the function. The final report
  \texttt{R} will be built as:
\begin{verbatim}
  <report name="R">
    <conjunction>
      <ref name = "M"/>
      <ref name = "F0" pos="0"/>
      <ref name = "F1" pos="1"/>
    </conjunction>
  </report>
\end{verbatim}
  Please note that the positions given here refer to the positions in the result
  row.  It is not the position of the argument in the call of the method
  analyzed.
\end{itemize}

\subsubsection{Structure element}
This is a feature introduced with Android profile. The structure element
describes how the report is organized. It contains an HTML body where some tags
specify where to put the result of a given analysis. The tags are
\verb!<callRef name="..."/>! and \verb!<fieldRef name="..."/>! where the
attributes is the name of the rule (not the report) to inline at this point.
It can also contain \verb!<sdiv>! elements. An SDIV is a division that is
displayed only if there is a content in it: at least one of the rules that is
specified with callRef or fieldRef has a result to display. An sdiv element
may be named (optional \verb!name! attribute). Finally the \verb!<nodes/>!
element is used to declare the location where the node table must be inlined.

Note that the Android profile also contains some JavaScript to make it possible
to have foldable explanations. The JavaScript is tailored to handle two
languages. Contents must be defined in contiguous sections (using a \verb!div!
element) whose class are respectively \verb!AuxFoldBox! and \verb!FoldBox!).

\subsubsection{The score section}
The \verb!score! tag introduces a basic scoring mechanism for Anasoot. The score
section contains score elements that describes the individual contributions and
text that is only used in the documentation of the profile. The score element
has one attribute \verb!threshold! defining the value above which the analysis
is considered as failed.

Each score element has a name (attribute \verb!name!), a message (attribute
\verb!message!) and a score value (attribute \verb:score:). The name is used to
refer to the score element in other elements. The message is the text displayed
in the report. The score is the contribution of the element to the global score.
If it is 0 (no contribution), the element is NOT displayed. But it is still
marked as triggered and may be used by other elements (any-of and all-of).

The tags used to describe score elements are the following :
\begin{itemize}
  \item \verb!stringmatch! matches a string in the report. The element uses a
  \verb!pattern! attribute describing a regular expression (in Java regexp
  format with special characters escaped in XML format). Do not use patterns
  spanning over multiple results as in fact the search is local to an element.
  \item \verb!rulematch! element contains a list of \verb!rule! element
  characterized by a \verb!name! attribute. This element is triggred if any of
  the rules has been triggered during the analysis. A rule name is either an
  Anasoot rule or the name of an \verb!sdiv! section grouping several rules. The
  section is considered as trigered if any of the rules it contains has been
  trigered.
  \item \verb!any-of! is trigered if any of the score elements it refers to has
  been trigered. Score element are identified by the \verb!element! tags. These
  tags contain a \verb!name! attribute.
  \item \verb!all-of! is trigered if all of the score elements it refers to have
  been trigered. It is similar to \verb!any-of!
  \item \verb!permission! matches a permission in the manifest. The full name of
  the permission must be given in the \verb:perm: attribute.
  \item \verb!catcher! are special elements. They have no score and no name.
  They are used to print the list of elements in the report matching the
  \verb!pattern! attribute. The \verb!message! attribute is the text printed as
  header for this section.
\end{itemize}


\subsubsection{Score element}


\subsubsection{Global structure of HTML report}
The report should be a compliant XHTML document. So that automatic XSL
transforms can be done on the document, we have introduced a few division in the
generated report whose class attribute can be used to select the division
content.

The following class are used in the report:
\begin{itemize}
  \item header : the header part of the report (MIDP or Android)
  \item compatibility : indication on hidden APIs used in an Android report
  \item score : the score and how the score is split between subelements
  \item catched : the substring that are isolated at the end (for example URLs)
  \item AndroidManifest : the manifest analysis
  \item android : main analysis.
\end{itemize}

\subsection{Customization of descriptors conformity phase}
\label{CustoConform}

To add properties to respect for the descriptors conformity phase, options
described in section \ref{optionsSecurityProfile} are used. You can add
some properties of JAD file and/or JAR manifest file about:
\begin{itemize}
  \item Line end characters.
  \item Order of attributes.
  \item Mandatory attributes.
  \item Forbidden attributes. 
\end{itemize}

\subsubsection{Line end separator}
According to the operating system, the line end character could be only
\texttt{CR LF} (\verb!\r\n!), \texttt{CR} 
(\verb!\r!) or \verb!LF (\n)!. The MIDP 2.0 specification
authorizes only to use \texttt{CR LF} or \texttt{LF} character as end of line in
the JAD file. You can specify only a subset of these characters as
authorized line end separator for the descriptors. The given characters must be
separated by a comma.\\
Because the security profile is written in XML, the line end characters must be
coded like indicated below:
\begin{itemize}
  \item{CR LF}: \verb!&#x000D;&#x000A;!
  \item{CR}: \verb!&#x000D;!
  \item{LF}: \verb!&#x000A;!
\end{itemize}

For instance,
\begin{verbatim}
  <option name="descriptor.jad.endline" value="&#x000D;&#x000A;">
    <!-- This option defines that the only authorized character to 
         mark the end of line is CR LF -->
    <description>
      This option gives authorized characters to mark the end of line 
	  in the JAD file.
    </description>
  </option>
  <option name="descriptor.jar.endline" 
          value="&#x000D;&#x000A;,&#x000A;">
    <!-- This option defines that authorized characters are CR LF 
         and LF -->
    <description>
      This option gives authorized characters to mark the end of line 
      in the JAR manifest file.
    </description>
  </option>
\end{verbatim}

\subsubsection{Order of attributes}

You can define an order of attributes to respect in JAD file and/or in JAR
manifest file. The \ma will notify an error in the report if the attributes' 
order does not respect the order specified in the security profile. To specify
this order, there are two options: \emph{descriptor.jad.attributesorder} and
\emph{descriptor.jar.attributesorder}. By default, no order is specified.

Syntax by the example:
\begin{verbatim}
  <option name="descriptor.jad.attributesorder" 
          value="MIDlet-Jar-URL,MIDlet-Jar-Size">
    <!-- This option declares that MIDlet-Jar-URL attribute must appear
         before MIDlet-Jar-Size attribute in JAD file -->
    <description>
      This option controls attributes' order in JAD file.
    </description>
  </option>
  <option name="descriptor.jar.attributesorder" 
          value="MIDlet-Version,MIDlet-Name">
    <description>
      This option controls attributes' order in JAR manifest file.
    </description>
  </option>
\end{verbatim}

If a given attribute does not exist in the file, it is not taken into account.

\subsubsection{Mandatory attributes}
Likewise, to define attributes that must be appear in JAD file or in JAR
manifest file, there are two options: \emph{descriptor.jad.mandatory} and
\emph{descriptor.jar.mandatory}. 

For instance,
\begin{verbatim}
  <option name="descriptor.jar.mandatory"
          value="Created-By">
    <!-- This option declares that Created-By attribute is mandatory
         in JAR manifest file -->
      <description>
        This option gives attributes which must appear in JAR manifest 
 		file.
      </description>
  </option>
\end{verbatim}

By default, mandatory attributes are only mandatory attributes specified by the
MIDP 2.0 specification.

\subsubsection{Forbidden attributes}
In the same way, you can specify which must not appear in JAD file or in JAR
manifest file. By default, no forbidden attributes are specified.

For instance,
\begin{verbatim}
  <option name="descriptor.jad.forbidden"
          value="MIDlet-Jar-RSA-SHA1">
    <!-- This option declares that MIDlet-Jar-RSA-SHA1 attribute must 
         not appear in JAD file -->
      <description>
        This option gives attributes which are forbidden in JAD file.
      </description>
  </option>
\end{verbatim}

\subsection{Syntax For Regular Expressions (Filter Patterns)}
Regular expressions that you can use in filters follow the syntax of Sun regular
expressions (closely related to POSIX syntax of regular expressions).
\begin{description}
\item[Characters]
%TODO: ALIGNEMENT A AMELIORER, PAS TRES lisible tel quel
   \mbox{}\begin{itemize}
    \item \textit{unicodeChar} filters the corresponding Unicode character
    \item \verb!\\\\!        backslash
    \item \verb!\\0nnn!      character expressed in octal code
    \item \verb!\\xhh!       character expressed in hexadecimal code
    \item \verb!\\uhhhh!     16-bit character expressed in hexadecimal code
    \item \verb!\t!          tabulation
    \item \verb!\n!          new line
    \item \verb!\r!          carriage return
    \item \verb!\f!          form feed
   \end{itemize}
\item[Character classes (sets)]
    \mbox{}\begin{itemize}
    \item \verb![abc]!                one character among a, b and c
    \item \verb![a-zA-Z]!             one character from the a-z range or from the A-Z range
    \item \verb![^abc]!               complement: one character different from a, b, and c
    \end{itemize}
\item[POSIX character classes]
    \mbox{}\begin{itemize}
    \item \verb![:alnum:]!            alphanumeric character
    \item \verb![:alpha:]!            alphabetical character
    \item \verb![:blank:]!            space or tab
    \item \verb![:cntrl:]!            control character
    \item \verb![:digit:]!            numeric character
    \item \verb![:graph:]!            displayable and visible character
    \item \verb![:lower:]!            lower-case character
    \item \verb![:print:]!            non-control character
    \item \verb![:punct:]!            ponctuation character
    \item \verb![:space:]!            space
    \item \verb![:upper:]!            upper-case character
    \item \verb![:xdigit:]!           hexadecimal digit
    \end{itemize}
\item[Non-standard classes]
    \mbox{}\begin{itemize}
    \item \verb![:javastart:]!        begin of a Java identifier
    \item \verb![:javapart:]!         part of a Java identifier
    \end{itemize}
\item[Predefined classes]
    \mbox{}\begin{itemize}
    \item \verb!.!   any character except new line
    \item \verb!\\w! a word character (alphanumeric or '\_') %TODO: c quoi ca ???
    \item \verb!\\W! anything but a word character
    \item \verb!\\s! white space
    \item \verb!\\S! anything but a white space
    \item \verb!\\d! digit
    \item \verb!\\D! anything but a digit
    \end{itemize}
\item[Bound filters]
    \mbox{}\begin{itemize}
    \item \verb!^!         beginning of a line
    \item \verb!$!         end of a line (%$)
    \item \verb!\\b!       filters within the limit of a word
    \item \verb!\\B!       filters beyond the limit of a word
    \end{itemize}
\item[Greedy closures] %TODO: Je sais pas traduire fermeture AVIDE
    \mbox{}\begin{itemize}
    \item \verb!A*!        any number of the \verb!A! character
    \item \verb!A+!        at least one \verb!A!
    \item \verb!A?!        zero or one \verb!A!
    \item \verb!A{n}!      exactly $n$ occurrences of \verb!A!
    \item \verb!A{n,}!     at least $n$ occurrences of \verb!A!
    \item \verb!A{n,m}!    at least $n$ occurrences of \verb!A! but no more than
    $m$
    \end{itemize}
\item[Reluctant closures]
    \mbox{}\begin{itemize}
    \item \verb!A*?!        any number of \verb!A!
    \item \verb!A+?!        at least one \verb!A!
    \item \verb!A??!        zero or one \verb!A!
    \end{itemize}
\item[Logical operators]
    \mbox{}\begin{itemize}
    \item \verb!AB!        \verb!A! followed by \verb!B!
    \item \verb!A|B!       \verb!A! or \verb!B!
    \item \verb!(A)!       grouping of sub-expressions
    \item \verb!(?:A)!     clustering of sub-expressions (just like grouping but no backrefs)
%TODO: Pas sur de la traduction pour ``groupage de sous expression mais n'apparaissant pas en r�f�rence arri�re''
    \end{itemize}
\item[Rear reference]
    \mbox{}\begin{itemize}
    \item \verb!\\i! reference to the sub-expression number i (from 1 to 9)
    \end{itemize}
\end{description}

You can make references to information of the analysis results from your messages.
\begin{itemize}
\item \verb!%r! the entire filtered result (string)
\item \verb!%m! the method enclosing the analysed expression
\item \verb!%c! the class that defines the method enclosing the analysed expression
\item \verb!%s! the signature of the method enclosing the analysed expression
\item \verb!%M! the analysed (probed) method
\item \verb!%C! the class that defines the analysed (probed) method
\item \verb!%S! the signature of the analysed method
\item \verb!%%! the \% character
\item \verb!%1!....\verb!%9! reference to a named sub-expression of a filter
 %TODO: detailler un peu tout cela
\end{itemize}


\subsection{Syntax Formal Specification (DTD)}
Analysis profiles are XML files that must follow the DTD presented in
the present section.
This DTD is stored in the \texttt{matos.dtd} file,
located in the \texttt{lib} folder of your installation directory
(\texttt{lib/matos} folder on Linux).
\begin{verbatim}
<!ELEMENT matos (option|report|rule)*>
<!ATTLIST matos
    name CDATA #REQUIRED
    description CDATA #IMPLIED
    version CDATA #IMPLIED
    date CDATA #IMPLIED
>
<!ELEMENT option (description?)>
<!ATTLIST option
    name CDATA #REQUIRED
    value CDATA #REQUIRED
>

<!ELEMENT report
  ((conjunction|pseudoConstant|pseudoString|message|forbiddenMethod),
  description?)>
<!ATTLIST report name CDATA #REQUIRED
>

<!ELEMENT pseudoConstant (pass,fail)>
<!ELEMENT pass (%content;)>
<!ELEMENT fail (%content;)>

<!ELEMENT pseudoString (filter*,default?)>
<!ELEMENT filter (%content;)>
<!ATTLIST filter
    pattern CDATA #REQUIRED
>

<!ELEMENT default (%content;)>

<!ELEMENT forbiddenMethod (%content;)>

<!ELEMENT message (%content;)>

<!ELEMENT conjunction (ref)*>

<!ELEMENT ref EMPTY>
<!ATTLIST ref
    name CDATA #REQUIRED
    pos  CDATA  
>

<!ELEMENT rule ((args|return|field|use), description?)>
<!ATTLIST rule
   name  CDATA #REQUIRED
>

<!ELEMENT args (argument)*>
<!ATTLIST args
   class CDATA #REQUIRED
   signature CDATA #REQUIRED
   report CDATA "terse"
>
<!ELEMENT argument EMPTY>
<!ATTLIST argument
  position CDATA #REQUIRED
>

<!ELEMENT field EMPTY>
<!ATTLIST field
   class CDATA #REQUIRED
   type CDATA #REQUIRED
>

<!ELEMENT return EMPTY>
<!ATTLIST return
   class CDATA #REQUIRED
   signature CDATA #REQUIRED
>

<!ELEMENT use EMPTY>
<!ATTLIST use
   class CDATA #REQUIRED
   signature CDATA #REQUIRED
>

<!ELEMENT description (%content;)>

<!ELEMENT loop (translation|callback|critical)*>
<!ELEMENT translation (caller, callee, through*)>
<!ATTLIST translation
    index	CDATA	#IMPLIED
    message	CDATA	#IMPLIED
>

<!ELEMENT caller EMPTY>
<!ATTLIST caller
     class	CDATA   #IMPLIED
     signature	CDATA   #IMPLIED
>

<!ELEMENT through EMPTY>
<!ATTLIST through
     class	CDATA   #IMPLIED
     signature	CDATA   #IMPLIED
     from	CDATA   #IMPLIED
     to 	CDATA   #IMPLIED
>

<!ELEMENT callee EMPTY>
<!ATTLIST callee
     signature        CDATA   #IMPLIED
>

<!ELEMENT critical EMPTY>
<!ATTLIST critical 
     class      CDATA   #IMPLIED
     signature  CDATA   #IMPLIED
     message 	CDATA   #IMPLIED
>

<!ELEMENT callback EMPTY>
<!ATTLIST callback 
     class       CDATA   #IMPLIED
     signature   CDATA   #IMPLIED
     message     CDATA  #IMPLIED
>

<!ELEMENT structure (callref,sdiv,fieldref,nodes,%content;)*>
<!ELEMENT nodes EMPTY>
<!ELEMENT callref EMPTY>
<!ATTLIST callref
	name	CDATA #REQUIRED
>

<!ELEMENT fieldref EMPTY>
<!ATTLIST fieldref
	name	CDATA #REQUIRED
>

<!ELEMENT sdiv (callref,sdiv,fieldref,nodes,%content;)*>
<!ATTLIST sdiv
	name	CDATA #IMPLIED
>

<!ELEMENT score (stringmatch,rulematch,any-of,all-of,permission,catcher,%content;)>
<!ATTLIST score 
     threshold       CDATA   #REQUIRED
>

<!ELEMENT stringmatch EMPTY>
<!ATTLIST stringmatch
	name	CDATA #REQUIRED
	score	CDATA #IMPLIED
	message CDATA #IMPLIED
	pattern CDATA #REQUIRED
>

<!ELEMENT rulematch (element)*>
<!ATTLIST rulematch
	name	CDATA #REQUIRED
	score	CDATA #IMPLIED
	message CDATA #IMPLIED
>

<!ELEMENT element EMPTY>
<!ATTLIST element
	name	CDATA #REQUIRED
>

<!ELEMENT any-of (element)*>
<!ATTLIST any-of
	name	CDATA #REQUIRED
	score	CDATA #IMPLIED
	message CDATA #IMPLIED
>

<!ELEMENT all-of (element)*>
<!ATTLIST all-of
	name	CDATA #REQUIRED
	score	CDATA #IMPLIED
	message CDATA #IMPLIED
>

<!ELEMENT permission EMPTY>
<!ATTLIST permission
	name	CDATA #REQUIRED
	id		CDATA #REQUIRED
	score	CDATA #IMPLIED
	message CDATA #IMPLIED
>

<!ELEMENT catcher EMPTY>
<!ATTLIST catcher
	message	CDATA #REQUIRED
	pattern	CDATA #REQUIRED
>
\end{verbatim}

%=============

\section{Configuration Options}\label{AdvConfigOptions}
The tool's behaviour is influenced by various configuration
options. These are declared in the \texttt{config.prp} file located
in the \texttt{lib} folder of your installation directory on Windows
(\texttt{lib/matos} on Linux). In fact, you should
normally seldom change these parameters. The available options are listed
below. Remember that the very essential ones are those editable
from the GUI Configuration panel.

\noindent \textbf{Note:} Boolean properties are considered false only if the
exact value is \texttt{\textbf{false}}. If the property is defined with another
value, or not defined at all, then it is considered \textbf{true}.
The \texttt{anasoot} prefix designates attributes specific
to the Features Usage phase, for historical reasons.

%TODO : verifier completude !!! Publier pas forcement tout. Pas ce qui
% concerne les modules protos comme Navmid en tout cas.

\begin{itemize}
\item{\texttt{anasoot.enabled}}\\ activate the features usage analysis phase 
  (boolean: \texttt{true} or \texttt{false}).
\item{\texttt{anasoot.android.global}}\\ makes a global analysis rather than a
per component analysis of Android applications (boolean: \texttt{true} or
\texttt{false}, the recommended default is \texttt{true}).
\item{\texttt{anasoot.enhancedPointsto}}\\ use an enhanced pointsto analysis
that assign new objects to the results of method of the analysed profile. It
creates more abstract objects and so gives more precise results (boolean:
\texttt{true} or \texttt{false}).
\item{\texttt{anasoot.reportPseudo}}\\ activate the reporting of pseudo
  constants (boolean: \texttt{true} or \texttt{false}).
\item {\texttt{anasoot.loopAnalysis}}\\ performs an analyis of recursion for
profiles that forbid the use of recursion (boolean: \texttt{true} or
\texttt{false}).
\item{\texttt{anasoot.ruleDefaultFile}}\\ name of the default analysis profile
  (file name without the extension).
\item{\texttt{anasoot.treatConcatenation}}\\ enable treatment of appended
  strings (boolean).
\item{\texttt{anasoot.treatStringBuffer}}\\ enable treatment of basic operations on
  \texttt{StringBuffer} objects, a special kind of string in Java (boolean).
\item{\texttt{anasoot.treatValueOf}}\\ enable treatment of
  \texttt{String.valueOf} (boolean).
\item{\texttt{anasoot.treatProperties}}\\ enable treatment of
  \texttt{getAppProperty}, with link to the JAD (boolean).
\item{\texttt{anasoot.treatStringOperation}}\\ enable treatment of basic
  operations on String objects (\texttt{trim}, \texttt{replace}, etc.) (boolean).
\item{\texttt{anasoot.treatFunctions}}\\ enable treatment of functions in
  general (followed by name and parameters) (boolean).
\item{\texttt{anasoot.treatSpyReturn}}\\ enable treatment of values returned by
  user's methods (boolean).
\item{\texttt{anasoot.treatSpyParameter}}\\ enable treatment of arguments passed
  to user's methods (boolean).
\item{\texttt{anasoot.treatSpyField}}\\ enable treatment of user's class
  attributes (boolean).
\item{\texttt{anasoot.printNodes}}\\ enable useless mentions of node numbers in
  report (boolean).
%\item{\texttt{anasoot.reportPseudo}}\\
%\item{\texttt{anasoot.callbackAnalysis}}\\ activate callbacks analysis phase to
% to piece together the navigation between screens in a MIDlet.
%\item{\texttt{thali.enabled}} \\ activate the THALI phase.
\item{\texttt{anasoot.midp-classpath}}\\ path to libraries which are used in the
  features usage analysis phase.
\item{\texttt{anasoot.rules}}\\ directory which contains analysis profiles.
%\item{\texttt{anasoot.throwCode}} enable to generate a file which contains
%``soot'' outputs.
\item{\texttt{anasoot.loopAnalysis}}\\ enable (default) or disable the analysis
that points out recursive method
\item{\texttt{anasoot.usedJSR}}\\ enable (default) or disable the analysis
that points out the use of a set of MIDP JSR. 
\item{\texttt{descriptorsChecking.enabled}}\\ enable the descriptors conformity check phase (boolean).
\item{\texttt{proxyset}}\\ enable use of the HTTP proxy (boolean).
\item{\texttt{proxyhost}}\\ hostname of HTTP proxy (if needed, for downloading of
  MIDlets that are accessible on a web server).
\item{\texttt{proxyport}}\\ port number of HTTP proxy. 
\item{\texttt{httpMaxConnectionTime}}\\ maximum connection time allowed for HTTP
  downloads (in seconds).
\item{\texttt{httpMaxDownloadTime}}\\ maximum transfer time allowed for HTTP
  downloads (once the connection has succeeded, and in seconds).
\item{\texttt{version}}\\ current version of the \ma.
\item{\texttt{matos.cssFile}}\\ style sheet (CSS file) to use for the HTML
  reports (path to a CSS file).
\item{\texttt{matos.history}}\\ enable the storage of all results if results are
  stored in a directory (boolean).
\item{\texttt{matos.considerSameName}}\\ enable to recognize that a JAD file and a JAR
  file are relative to the same MIDlet suite if they have the same name, when
  you add all MIDlet suite of a directory (boolean). 
\ifthenelse{\equal{\Gallery}{true}}{
\item{\texttt{matos.resultStorageMethod}}\\ one of these options: 
	\begin{itemize}
      \item{\emph{ArchivingDBResultStorage}} analysis results are stored in a
      database.
      \item{\emph{DirectoryResultStorage}} analysis results are stored in a
      directory.
      \item{\emph{NoResultStorage}} analysis results are not saved.
    \end{itemize} 
\item{\texttt{matos.allowRetrievingPreviousResults}}\\ enable to retrieve
  previous results from database (boolean).
\item{\texttt{matos.retrieveForLocalStepOnly}}\\ enable to retrieve previous
  results from database uniquely for JAD and JAR files which are on the hard
  disk (boolean). 
}{
\item{\texttt{matos.resultStorageMethod}}\\ one of these options: 
	\begin{itemize}
      \item{\emph{DirectoryResultStorage}} analysis results are saved in a
      directory.
      \item{\emph{NoResultStorage}} analysis results are not saved.
    \end{itemize} 
}
\item{\texttt{matos.pdf.encryption}}\\ enable to put permission restrictions on
generated PDF files (boolean).
\item{\texttt{matos.pdf.encryption.userpasswd}}\\ the opening of generated PDF
files will be protected with the given password (password).
\item{\texttt{matos.pdf.encryption.ownerpasswd}}\\ a password to bolt some
actions, the only action that is always authorized without password is to print
the document (password).
\item{\texttt{matos.xmlFormat}}\\ allow XML formatting of results (boolean). 
\ifthenelse{\equal{\Gallery}{true}}{
\item{\texttt{matos.dbUrl}}\\ url of used database.
\item{\texttt{matos.dbLogin}}\\ login of database user.
\item{\texttt{matos.dbPasswd}}\\ password of database user.
\item{\texttt{matos.cid.mainSheetName}}\\ the main sheet name of Excel files
which contains MIDlet suites of editor.
}{}
\item{\texttt{descriptor.jad.endline}}\\ comma separated list of authorized
 characters to mark the end of line in the JAD file.
\item{\texttt{descriptor.jar.endline}}\\gives authorized characters to mark the
 end of line in the JAR manifest file.
\item{\texttt{descriptor.mandatorChapiId}}\\boolean value (default false)
 specifying if JSR 211 content-handler must have an explicit ID
\item{\texttt{descriptor.chapiContentTypesRegexp}}\\optional regular expression
 defining constraints on the each content-type mentioned in a JSR 211 handler 
\item{\texttt{descriptor.chapiActionsRegexp}}\\optional regular expression
 defining constraints on the each action mentioned in a JSR 211 handler
\item{\texttt{descriptor.chapiSuffixesRegexp}}\\optional regular expression
 defining constraints on the each file suffix mentioned in a JSR 211 handler
\item{\texttt{descriptor.midletSuiteNameRegexp}}\\optional regular expression
 defining constraints on the name of the midlet suite
\item{\texttt{descriptor.vendorNameRegexp}}\\optional regular expression
 defining constraints on the name of the vendor of the midlet suite
\item{\texttt{descriptor.jad.attributesorder}}\\comma separated list defining
 the order of jad attributes (useful for some download portal that cannot
 handler attributes in random order)
\item{\texttt{descriptor.jar.attributesorder}}\\comma separated list defining
 the order of jar attributes (useful for some download portal that cannot
 handler attributes in random order)
\item{\texttt{descriptor.jad.forbidden}}\\ additional list of attributes
 forbidden in the JAD
\item{\texttt{descriptor.jad.mandatory}}\\ additional list of attributes 
 mandatory in the JAD
\item{\texttt{descriptor.jar.mandatory}}\\ additional list of attributes
 forbidden in the JAR manifest
\item{\texttt{descriptor.jar.forbidden}}\\ additional list of attributes
 mandatory in the JAR manifest
\item{\texttt{descriptor.midletSuiteIconSize}}\\optional comma delimited list of
 authorized sizes for the midlet suite icons (as \textit{width}x\textit{height})
\item{\texttt{descriptor.maxMidlets}}\\ Optional constraint defining the maximum
 number of visible services (midlets) defined by the suite.
\item{\texttt{descriptor.midletNameRegexp}}\\optional regular expression
defining constraints on the names of midlets (services) provided by the suite.
\item{\texttt{descriptor.midletIconSize}}\\optional comma delimited list of
authorized sizes for midlet icons (as \textit{width}x\textit{height})
\item{\texttt{descriptor.pushURLRegexp}}\\optional regular expression
defining constraints on the URL defined for static push registry
\item{\texttt{descriptor.pushRestrictionRegexp}}\\optional regular expression
defining constraints on the restrictions imposed for static push registry.
\end{itemize}



%TODO: a documenter plus tard (DTD des fichiers de campagne .mcl)
%section{The Check-list File Format}


%%% Local Variables: 
%%% mode: latex
%%% TeX-master: t
%%% End: 
