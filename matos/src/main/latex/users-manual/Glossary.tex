\chapter*{Glossary}
\addcontentsline{toc}{chapter}{Glossary}

\paragraph{Analysis process} the work performed by the \ma on a given
MIDlet suite. Consists of a series of phases.
\paragraph{Analysis profile} a special customisation of the tool
enumerating a set of analysis directives, and the associated reporting
rules. An analysis profile is an implementation of a security policy.
\paragraph{Check-list} a sequence of steps.  A check-list for the \ma
is similar to a play-list for an audio player. 
\paragraph{MIDlet} a Java application developed for the MIDP environment.
\paragraph{MIDlet suite} a bunch of MIDlets, materialized by a JAR
file and described by an associated JAD file.
\paragraph{Phase} one part (or stage) of the analysis
process. Currently, only two phases make up the whole process: the
first phase controls the conformity of descriptors, and the second
phase studies the usage of particular Java features (MIDlet by MIDlet).
\paragraph{Security policy} a set of requirements related to security.
\paragraph{Step} a (\emph{check-})step represents one execution of the
analysis process: it specifies how the analysis process
should be run (on what MIDlet suite, with what parameters, etc.) The
step is the basic unit of a check-list, and for the \ma, it is similar
to a song for an audio player.

%%% Local Variables: 
%%% mode: latex
%%% TeX-master: "Users-manual"
%%% End: 
