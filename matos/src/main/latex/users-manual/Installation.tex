\chapter{Installation}


Make sure you have a Java Runtime Environment installed first
(cf. System Requirements, page \pageref{SysReqs}).

\section{Windows Version}
Run the \texttt{setup.exe} file provided in your \ma distribution. The
installer works as a very classical Windows application installer,
just keep all default settings and press \texttt{Next} twice then
\texttt{Install}. A more detailed explanation follows:
\begin{itemize}
\item The first screen lets you select if you want to install the menu shortcuts in the main Startup menu. The answer is usually yes
\item The second screen lets you select the default destination folder. Consider the next warning if you are not using Windows XP
%\item The third screen lets you define a folder where the results of your
analysis will be stored. By default it is in the matos folder.
\item The last screen installs the tool. The complete installation requires around 5Mbytes.
\item An ``uninstaller'' is provided with the tool.
\end{itemize}

Warning: we cannot guarantee that this software is compatible with either
Windows Vista or Windows 7. Nevetheless, it is written as a pure Java
application and should work on those newer operating systems.

%\paragraph{Warning if you are not using XP} 
% \begin{itemize}
% \item Space characters in paths are badly handled. So, you must install the
%   \ma at the root level of the C: volume for example, rather than in
%   the traditional \texttt{Program Files} directory.
% \item The \texttt{matos.wsf} script is the one used to
% start the tool in interactive mode. This launcher makes use of the
% VBScript language, for which Windows XP has a built-in interpretor;
% for the NT4 version however, install the scripting languages package from
% Microsoft called \emph{Windows Script}. Check-out
% http://www.microsoft.com/downloads. The file you need is known as 
% \texttt{scr56en.exe}. 
% \end{itemize}

\subsection*{Warning about changes in your Java configuration}
If you install the \ma \emph{then later} change the installation directory
of your Java environment, the \ma won't find the new place by itself. It is
necessary to update the \texttt{matos.bat} launch script accordingly; this update
can be done automatically by executing the \texttt{initlauncher.wsf} script.

\section{Linux Version}
The software is provided as a kind of configurable and self-extractable archive
implemented as a shell-script. Open a terminal window and launch the
\texttt{./linux-installer} command in the directory where you put it.

The installer will perform the following step:
\begin{itemize}
\item it checks if all the shell commands it needs are available
\item it tries to find a java executable on your machine using simple 
heuristics
\item it proposes paths for the executable and the \ma library
\item it installs the software itself
\item it offers the possibility to configure a work repository where all the
results of analysis will be stored.
\end{itemize}

The installer tries to guess reasonable values for the installation paths. Just
type return if they are correct, otherwise, type in the right value.

Depending on whether you are logged as the super-user (root) or not, the 
installer will attempt a multi-user or a single user installation and
change the default settings.

The installation process has been tested on Mandrake/Mandriva and RedHat 
distributions. It relies on fairly standard tools and should work on 
other Linux distributions. Contact the support
team if you encounter any problem during the installation phase.

Make sure the \texttt{bin} folder of your installation directory
is included in your \texttt{PATH} environment variable, so that the
main command \texttt{matos} can be found.

\section{Adding Your License File} \label{AddingLicenseYourFile}

When you purchased the \ma, you received a license file. All you need to do
is to copy that file to the \texttt{license} folder that you'll find
in your installation directory on Windows (\texttt{lib/matos/license}
folder on Linux). To avoid problems when you update your version of the \ma, we
recommend that you keep a copy of that file in a safe place and never modify it. Contact the support
team if you encounter any problem related to the license.

\ifthenelse{\equal{\Gallery}{true}}{
\section{Creating database} \label{CreatingDatabase}
\textbf{If you do not plan to use this feature, please skip this section.}

There are different modes of operations of the \ma. In some of them, it is
useful to back-up your results in a real database. 
The \ma tool is able to use a SQL database to store CID files and to
manage the results  of analysis that have been performed on MIDlets.


The installation procedure is specific to the kind of database that will be used. 
We give here an example of a such installation for a MySQL \footnote{Although
the SQL commands issued by Coverity are pure standard SQL commands, we
recommend to use this database engine.} database, on a Windows system:

\begin{itemize}
  \item First, you need to install the database management system itself (if it
  is not already installed on the computer). An easy way to do so is to download
  and install EasyPHP (http://www.easyphp.org/).

  \item Second, you need to create the database, a user and the necessary
  tables that the \ma will use. To launch the database administration tool,
  rigth click on EasyPHP's tray icon and choose 'Administration', then click on 
  the 'Manage Database' link, and then click on the SQL icon on the left frame
  to open a query window. 
%  Click on 'Import file' and browse to select the file named
%  \texttt{initdb.sql} that comes with the installation files and click on 'Go'.  
  Select the 'mysql' database, copy paste the following mysql script into the text area and click on 'Go'
\end{itemize}
  \verbatiminput{initdb.txt} 

  By default, the database can only be accessed by the local computer.
  If you want it to be accessible from other computers, you have to:
  	\begin{itemize}
	    \item Set a root password (for security reasons):
		\begin{itemize}
			\item Launch the phpMyAdmin (rigth click on EasyPHP's tray icon and choose
			'Administration') and click on 'Manage Database' then on 'Privileges' and
			edit privileges of the root user. Set the password and click on the 'Go' 
			button.
			\item Edit the C:$\backslash$Program Files$\backslash$EasyPHP1.8$\backslash$phpmyadmin$\backslash$config.inc.php and
			put the new password for the root user:
			\newline \texttt{\$cfg['Servers'][\$i]['user'] = 'root';}
 			\newline \texttt{\$cfg['Servers'][\$i]['password'] ='the-new-password-4-root';} 
 			\newline Save the file.
	    \end{itemize}
	    \item Configure the database to allow distant access:
	    Rigth click on EasyPHP's tray icon and choose the menu entry
	    \texttt{Configuration->MySQL} to open the \texttt{my.ini} configuration
	    file and add a semi column (;) before the line defining the binding address
	    to comment it out:\newline
	    \texttt{bind-address=127.0.0.1} 
	    \newline Save the file.
	\end{itemize}

Use the following parameters to configure the \ma tool to connect to this
newly created database (using menu Tools->Databse Management
Tool->Edit->Database properties\ldots):
  	\begin{itemize}
        \item \texttt{\textbf{matos.dbUrl=jdbc:mysql://localhost/matos}} ,
        \item \texttt{\textbf{matos.dbLogin=matos\_user}} ,
	    \item \texttt{\textbf{matos.dbPasswd=matosPassword}} 
    \end{itemize}

}{}
%%% Local Variables: 
%%% mode: latex
%%% TeX-master: "Users-manual"
%%% End: 
