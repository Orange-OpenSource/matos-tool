\chapter*{Introduction}
\addcontentsline{toc}{chapter}{Introduction}


France Telecom's \ma is a framework to perform various analyses
on Android applications or MIDlet suite. 
Its main purpose is to check that mobile applications conform to a well
defined \emph{security policy}. Typically, a security policy may forbid the application to send short messages (SMS), open datagram or HTTP connections, or read a given system property. The \ma verifies whether the mobile application makes use (and how) of a  set of given
features. This is the purpose of the main phase of the overall
\emph{analysis process}, introduced as Features Usage phase below.
To achieve this, the \ma works on the main files that make up an
application: for example, for a MIDlet, the JAR, which contains the application
executable code, and the JAD, which is a readable descriptor of the application.

Though the analysis process is partly hard-coded in the tool, the \ma
is mainly guided by an \emph{analysis profile}, a document that
describes what is to be checked on the application. The analysis profile basically
defines the instructions to give to the \ma. 
There are mainly two kinds: what is to be observed, and
what reaction must be taken (in particular, what to report, pass or
fail, etc.) according to the observations done. 
When launching the analysis process, the user must specify which profile is
used for this analysis.

The analysis process is a sequence of \emph{phases} applied one after
another on the target application, each phase being specialized in a
specific treatment.

Currently, three phases are defined:
\begin{description}
  \item[The Class Identification phase:] that isolates midlets and checks that
  some system class are not reimplemented by the application and isolates some
  specific objects (as Views in Android).
  \item[The Descriptor Conformity phase:] checks the correctness of descriptors, namely 
  the JAD file, and the manifest file located into the JAR. This phase is
  applied at the level of the MIDlet suite. For Android applications, this phase
  extract some relevant information from the Android Manifest.
  \item[The Features Usage phase:] checks the way some
  well-identified Java features are used by the MIDlet(s) of the JAR or the
  components of the DEX file, by analysing what values can be taken by their
  arguments during all possible executions of the application. 
  This phase is applied to each MIDlet contained in the MIDlet suite but it is
  usually global for the components of the android application.
\end{description}

A \emph{MIDlet suite} is a Java application using the MIDP platform (with J2ME
CLDC configuration). It is composed of individual component (often only one)
called \emph{MIDlets}. An android application is made of different components
(namely activities, services, content providers and broadcast receivers) defined
in a single package known as the APK file (Android package). The APK file
contains an Android Manifest that is the equivalent of the JAD file, some
arbitrary resources and a code file known as the DEX file. It corresponds to the
JAR file.
%%% Local Variables: 
%%% mode: latex
%%% TeX-master: "Users-manual"
%%% End: 
